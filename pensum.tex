
\documentclass[a4paper,11pt]{article}
\usepackage[headheight=14pt, top=0.5cm, bottom=0.5cm, left=.5cm, right=.5cm]{geometry}
\usepackage{graphicx}
\usepackage{amsmath}
\usepackage{tikz}
\usepackage{float}
\usepackage{transparent}
\usepackage{xifthen}
\usepackage{import}
\usepackage[final]{pdfpages}
\def\checkmark{\tikz\fill[scale=0.4](0,.35) -- (.25,0) -- (1,.7) -- (.25,.15) -- cycle;}
\setboolean{@twoside}{false}
\pdfminorversion=7
\setlength\parindent{0pt}

\newcommand{\incfig}[1]{%
    \def\svgwidth{\columnwidth}
    \import{./figures/}{#1.pdf_tex}
}

\begin{document}
\title{5. semester pensum}
\date{}
\maketitle
\vspace{-2cm}

\tableofcontents
\section{Eksamener}

\begin{itemize}
    \item 02/12/2020 - Statistik
    \item 13/01/2020 - Algoritmer og datatrukturer
    \item 20/01/2020 - Rob,Vis,AI
    \item 21/01/2020 - Rob,Vis,Ai
    \item 22/01/2020 - Rob,Vis,Ai
    \item 31/01/2020 - Linux
\end{itemize}

\section{Kurser}

\subsection{Robotter i Kontekst}

    \begin{itemize}
        \item Terminology
            \begin{itemize}
                \item Optimality
                \item Computational Complexity
                    \begin{itemize}
                        \item Completeness
                            \begin{itemize}
                                \item Resolution
                                \item Probabilistic
                            \end{itemize}
                        \item Online / Offline
                        \item Sensor / Model based
                        \item Feedback / Blind
                        \item Bounded / unbounded configuration space
                        \item Holonomic
                        \item Configuration Space
                        \item Homogenous Transform
                        \item Sensors
                    \end{itemize}
            \end{itemize}
    \end{itemize}
    \begin{itemize}
        \item Motion algoritmer
            \begin{itemize}
                \item Bug 0
                \item Bug 1
                \item Bug 2
            \end{itemize}
    \end{itemize}
    \begin{itemize}
        \item Line representation
            \begin{itemize}
                \item Line fitting
            \end{itemize}
        \item Line extration algoritmer
            \begin{itemize}
                \item Spilt-and-merge (Slide 42 i summary)
                \item Incremental (Slide 43 i summary)
                \item RANSAC (Slide 44 i summary)
                \item Hough transform (Slide 45 i summary)
            \end{itemize}
        \item Localization
            \begin{itemize}
                \item Challenges i theory
                    \begin{itemize}
                        \item Noise
                        \item Aliasing
                    \end{itemize}
            \end{itemize}
        \item Potential fields
            \begin{itemize}
                \item fra slide 53 in summary 
            \end{itemize}
        \item BrushFire
        \item Wavefront
        \item Roadmaps
            \begin{itemize}
                \begin{itemize}
                    \item Accessibility (Slide 77, summary)
                    \item Departability (Slide 77, summary)
                    \item Connectivity  (Slide 77, summary)
                \end{itemize}
            \item Grid sampling
            \item Visibility Graph
            \item Voronoi Diagram
            \item cannys roadmap method
            \item Probabilistic Roadmaps
            \item Cell Decomposition
                \begin{itemize}
                    \item Trapezoidal decomp
                    \item Coverage planning
                    \item Boustrophedon decomp
                    \item Morse decomp
                    \item Incremental Coverage
                \end{itemize}
            \end{itemize}
    \end{itemize}

\subsection{AI}

\begin{itemize}
    \item Turing test
        \begin{itemize}
            \item Systems that act like humans
            \item Systems that think like humans
            \item Systems that think rationally
        \end{itemize}
    \item Rational
    \item Agent
        \begin{itemize}
            \item Reactive Agent
            \item Deliberative Agent
        \end{itemize}
    \item Environment 
        \begin{itemize}
            \item Accessibility?
            \item Deterministic?
            \item Episodic?
            \item Static?
            \item Discrete?
        \end{itemize}
            \item Typer af læring
        \begin{itemize}
            \item Supervised learning
            \item Unsupervised learning
            \item Reinforcement learning
        \end{itemize}
    \item Kategorier
        \begin{itemize}
            \item Strong AI
            \item Weak AI
        \end{itemize}
        \begin{itemize}
            \item Fuzzy logic
                \begin{itemize}
                    \item Crisp value
                    \item Degree of membership
                    \item Fuzzy set
                    \item Universe of discourse
                    \item membership function
                        \begin{itemize}
                            \item Linguistic Value
                            \item Linguistic variable
                            \item Linguistic rules
                        \end{itemize}
                        \begin{itemize}
                            \item Fuzzification 
                            \item T-norm
                            \item S-norm
                            \item Inference
                            \item Defuzzification
                                
                        \end{itemize}

                \end{itemize}
        \end{itemize}
        
\end{itemize}

\subsection{Computervision}
\subsection{Statistik}
\subsection{Linux}
\subsection{Algoritmer og datatrukturer}


\end{document}
